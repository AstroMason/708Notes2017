\documentclass{article}
\usepackage[utf8]{inputenc}
\usepackage{amsmath}
\title{Physics 708: Statistical Mechanics}
\author{Dion O'Neale 
	and someone else 
	and another person}
\begin{document}
\maketitle
\section{Lecture 1: Thermodynamics}
We'll start with a quick overview of some of the important concepts from classical thermodynamics. These ideas motivated much of statistical mechanics. This section will also recap some content from 315.

\subsection*{Macroscopic laws of thermodynamics}
In thermodynamics we study a system --- the part of the world that we are interested in --- that is separated from its surroundings --- the rest of the universe --- by some boundary.

We can classify the types of systems into three types based on the type of walls that define the system boundary:
\begin{itemize}
\item Adiabatic walls/isolated system - no energy or matter is transfered
\item Diathermal walls/closed system - no matter can be transfered but heat can flow through the walls.
\item (semi-)permeable walls/open system - in addition to heat, one or more chemical species can be transfered through the walls. 
\end{itemize}

\subsection*{The four laws of thermodynamics}
Only four laws are required to construct the relationships that control much of classical thermodynamics. Theses are, in brief:
\begin{itemize}
\item {\bf Zeroth law} Defines temperature and thermal equilibrium.
\item {\bf First law} Formulates the principle of conservation of energy for thermodynamic systems. (Energy is conserved)
\item {\bf Second law} Entropy increases; heat spontaneously flows from high to low temperatures.
\item {\bf Third law} The absolute zero of temperature is not attainable.  
\end{itemize}


We'll revisit the first two laws in a bit more detail and then make use of the second law to derive some familiar properties of heat.


Lecture one - a complete disaster - nothing here!
\end{document}
