\documentclass{article}
\usepackage[utf8]{inputenc}
\usepackage{amsmath}
\title{Physics 708: Statistical Mechanics}
\author{Dion O'Neale 
	and someone else 
	and another person}
\begin{document}
\maketitle
\section{Lecture 1: Thermodynamics}
We'll start with a quick overview of some of the important concepts from classical thermodynamics. These ideas motivated much of statistical mechanics. This section will also recap some content from 315.

\subsection*{Macroscopic laws of thermodynamics}
In thermodynamics we study a system --- the part of the world that we are interested in --- that is separated from its surroundings --- the rest of the universe --- by some boundary.

We can classify the types of systems into three types based on the type of walls that define the system boundary:
\begin{itemize}
\item Adiabatic walls/isolated system - no energy or matter is transfered
\item Diathermal walls/closed system - no matter can be transfered but heat can flow through the walls.
\item (semi-)permeable walls/open system - in addition to heat, one or more chemical species can be transfered through the walls. 
\end{itemize}

\subsection*{The four laws of thermodynamics}
Only four laws are required to construct the relationships that control much of classical thermodynamics. Theses are, in brief:
\begin{itemize}
\item {\bf Zeroth law} Defines temperature and thermal equilibrium.
\item {\bf First law} Formulates the principle of conservation of energy for thermodynamic systems. (Energy is conserved)
\item {\bf Second law} Entropy increases; heat spontaneously flows from high to low temperatures.
\item {\bf Third law} The absolute zero of temperature is not attainable.  
\end{itemize}


We'll revisit the first two laws in a bit more detail and then make use of the second law to derive some familiar properties of heat.

\subsection*{Zeroth law:}
If two systems are each in thermal equilibrium with a third system, then they are in thermal equilibrium with each other. This implies that they have some property in common. We call this property \emph{temperature}. (In the language of mathematics, thermal equilibrium is a transitive property.)

It is worth noting that thermal equilibrium is not the same as thermodynamic equilibrium. For the latter we also need mechanical equilibrium ($p_1=p_2$) and chemical equilibrium ($\mu_1=\mu_2$ --- equal rates of reaction).

\subsection*{First Law:}
Energy remains constant for a (collection of) system(s) isolated from the surroundings. When a system interacts with its surroundings any increase in the energy of the system is equal to the work done on the system by the surroundings. (We denote work done \emph{on} the system as $W>0$, similarly, heat supplied to the system is denoted $Q>0$.)

When considering the change in energy $\Delta E$ of a system it is necessary to consider both work and heat.

E.g. System $A$ gains energy from system $B$, i.e. $\Delta E_A = -\Delta E_B \implies \Delta E_A + \Delta E_B =0$. But, in general, $\Delta E_A\neq W_{B\rightarrow A}$ since there can also be a heat flow $Q_{B\rightarrow A}$ due to a temperature difference.
So,
$$\Delta E_A = W_{B\rightarrow A} + Q_{B\rightarrow A}$$
$$\Delta E_B = W_{A\rightarrow B} + Q_{A\rightarrow B}.$$

Energy conservation gives 
$$\underbrace{(W_{A\rightarrow B}+W_{B\rightarrow A})}_{\text{Work done by the composite system}} + \underbrace{(Q_{A\rightarrow B}+Q_{B\rightarrow A})}_{\text{Heat flow in the composite system}} = 0$$

In an adiabatic (isolated) system the fist law gives a sort of balance sheet for energy/work/heat:
$W_{A\rightarrow B}+W_{B\rightarrow A} = 0$ and $Q_{A\rightarrow B}+Q_{B\rightarrow A}$.

Revision: Adiabatic work and heat flow. Quasi-static processes (how slow is slow enough?). Path dependence of work.
 - see the notes from 315 that are in the repository. You may want to incorporate the relevant content from those notes, or other sources, into this document.


The \emph{internal energy} of a system is the energy associated with the internal degrees of freedom of the system. If a system is at rest and the potential energy of any external field is unimportant, then the internal energy of a system is the total energy of a system.

The equation of state for internal energy is usually written $E = E(T,V,N)$ or $E = E(T,P,N)$. Expressing these as infinitesimals (change in internal energy) gives
$$ dE = \frac{\partial E}{\partial T}\bigg\vert_{V,N}dT + \frac{\partial E}{\partial V}\bigg\vert_{T,N}dV +\frac{\partial E}{\partial N}\bigg\vert_{T,V}dN$$
a similarly, for the second formulation.

Note $ \frac{\partial E}{\partial T}\vert_{V,N} \neq \frac{\partial E}{\partial T}\vert_{P,N} $

\end{document}
